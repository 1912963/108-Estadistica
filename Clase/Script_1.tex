% Options for packages loaded elsewhere
\PassOptionsToPackage{unicode}{hyperref}
\PassOptionsToPackage{hyphens}{url}
%
\documentclass[
]{article}
\usepackage{lmodern}
\usepackage{amssymb,amsmath}
\usepackage{ifxetex,ifluatex}
\ifnum 0\ifxetex 1\fi\ifluatex 1\fi=0 % if pdftex
  \usepackage[T1]{fontenc}
  \usepackage[utf8]{inputenc}
  \usepackage{textcomp} % provide euro and other symbols
\else % if luatex or xetex
  \usepackage{unicode-math}
  \defaultfontfeatures{Scale=MatchLowercase}
  \defaultfontfeatures[\rmfamily]{Ligatures=TeX,Scale=1}
\fi
% Use upquote if available, for straight quotes in verbatim environments
\IfFileExists{upquote.sty}{\usepackage{upquote}}{}
\IfFileExists{microtype.sty}{% use microtype if available
  \usepackage[]{microtype}
  \UseMicrotypeSet[protrusion]{basicmath} % disable protrusion for tt fonts
}{}
\makeatletter
\@ifundefined{KOMAClassName}{% if non-KOMA class
  \IfFileExists{parskip.sty}{%
    \usepackage{parskip}
  }{% else
    \setlength{\parindent}{0pt}
    \setlength{\parskip}{6pt plus 2pt minus 1pt}}
}{% if KOMA class
  \KOMAoptions{parskip=half}}
\makeatother
\usepackage{xcolor}
\IfFileExists{xurl.sty}{\usepackage{xurl}}{} % add URL line breaks if available
\IfFileExists{bookmark.sty}{\usepackage{bookmark}}{\usepackage{hyperref}}
\hypersetup{
  pdftitle={Script\_1.R},
  pdfauthor={Usuario},
  hidelinks,
  pdfcreator={LaTeX via pandoc}}
\urlstyle{same} % disable monospaced font for URLs
\usepackage[margin=1in]{geometry}
\usepackage{color}
\usepackage{fancyvrb}
\newcommand{\VerbBar}{|}
\newcommand{\VERB}{\Verb[commandchars=\\\{\}]}
\DefineVerbatimEnvironment{Highlighting}{Verbatim}{commandchars=\\\{\}}
% Add ',fontsize=\small' for more characters per line
\usepackage{framed}
\definecolor{shadecolor}{RGB}{248,248,248}
\newenvironment{Shaded}{\begin{snugshade}}{\end{snugshade}}
\newcommand{\AlertTok}[1]{\textcolor[rgb]{0.94,0.16,0.16}{#1}}
\newcommand{\AnnotationTok}[1]{\textcolor[rgb]{0.56,0.35,0.01}{\textbf{\textit{#1}}}}
\newcommand{\AttributeTok}[1]{\textcolor[rgb]{0.77,0.63,0.00}{#1}}
\newcommand{\BaseNTok}[1]{\textcolor[rgb]{0.00,0.00,0.81}{#1}}
\newcommand{\BuiltInTok}[1]{#1}
\newcommand{\CharTok}[1]{\textcolor[rgb]{0.31,0.60,0.02}{#1}}
\newcommand{\CommentTok}[1]{\textcolor[rgb]{0.56,0.35,0.01}{\textit{#1}}}
\newcommand{\CommentVarTok}[1]{\textcolor[rgb]{0.56,0.35,0.01}{\textbf{\textit{#1}}}}
\newcommand{\ConstantTok}[1]{\textcolor[rgb]{0.00,0.00,0.00}{#1}}
\newcommand{\ControlFlowTok}[1]{\textcolor[rgb]{0.13,0.29,0.53}{\textbf{#1}}}
\newcommand{\DataTypeTok}[1]{\textcolor[rgb]{0.13,0.29,0.53}{#1}}
\newcommand{\DecValTok}[1]{\textcolor[rgb]{0.00,0.00,0.81}{#1}}
\newcommand{\DocumentationTok}[1]{\textcolor[rgb]{0.56,0.35,0.01}{\textbf{\textit{#1}}}}
\newcommand{\ErrorTok}[1]{\textcolor[rgb]{0.64,0.00,0.00}{\textbf{#1}}}
\newcommand{\ExtensionTok}[1]{#1}
\newcommand{\FloatTok}[1]{\textcolor[rgb]{0.00,0.00,0.81}{#1}}
\newcommand{\FunctionTok}[1]{\textcolor[rgb]{0.00,0.00,0.00}{#1}}
\newcommand{\ImportTok}[1]{#1}
\newcommand{\InformationTok}[1]{\textcolor[rgb]{0.56,0.35,0.01}{\textbf{\textit{#1}}}}
\newcommand{\KeywordTok}[1]{\textcolor[rgb]{0.13,0.29,0.53}{\textbf{#1}}}
\newcommand{\NormalTok}[1]{#1}
\newcommand{\OperatorTok}[1]{\textcolor[rgb]{0.81,0.36,0.00}{\textbf{#1}}}
\newcommand{\OtherTok}[1]{\textcolor[rgb]{0.56,0.35,0.01}{#1}}
\newcommand{\PreprocessorTok}[1]{\textcolor[rgb]{0.56,0.35,0.01}{\textit{#1}}}
\newcommand{\RegionMarkerTok}[1]{#1}
\newcommand{\SpecialCharTok}[1]{\textcolor[rgb]{0.00,0.00,0.00}{#1}}
\newcommand{\SpecialStringTok}[1]{\textcolor[rgb]{0.31,0.60,0.02}{#1}}
\newcommand{\StringTok}[1]{\textcolor[rgb]{0.31,0.60,0.02}{#1}}
\newcommand{\VariableTok}[1]{\textcolor[rgb]{0.00,0.00,0.00}{#1}}
\newcommand{\VerbatimStringTok}[1]{\textcolor[rgb]{0.31,0.60,0.02}{#1}}
\newcommand{\WarningTok}[1]{\textcolor[rgb]{0.56,0.35,0.01}{\textbf{\textit{#1}}}}
\usepackage{graphicx,grffile}
\makeatletter
\def\maxwidth{\ifdim\Gin@nat@width>\linewidth\linewidth\else\Gin@nat@width\fi}
\def\maxheight{\ifdim\Gin@nat@height>\textheight\textheight\else\Gin@nat@height\fi}
\makeatother
% Scale images if necessary, so that they will not overflow the page
% margins by default, and it is still possible to overwrite the defaults
% using explicit options in \includegraphics[width, height, ...]{}
\setkeys{Gin}{width=\maxwidth,height=\maxheight,keepaspectratio}
% Set default figure placement to htbp
\makeatletter
\def\fps@figure{htbp}
\makeatother
\setlength{\emergencystretch}{3em} % prevent overfull lines
\providecommand{\tightlist}{%
  \setlength{\itemsep}{0pt}\setlength{\parskip}{0pt}}
\setcounter{secnumdepth}{-\maxdimen} % remove section numbering

\title{Script\_1.R}
\author{Usuario}
\date{2020-01-30}

\begin{document}
\maketitle

\begin{Shaded}
\begin{Highlighting}[]
\CommentTok{# Adriana Concepción Garza Pérez}
\CommentTok{# Matrícula: 1912963}
\CommentTok{# Fecha: 29.01.2020}

\CommentTok{# Operadores básicos ------------------------------------------------------}

\DecValTok{2}\OperatorTok{+}\DecValTok{2}
\end{Highlighting}
\end{Shaded}

\begin{verbatim}
## [1] 4
\end{verbatim}

\begin{Shaded}
\begin{Highlighting}[]
\NormalTok{a<-}\DecValTok{2}
\NormalTok{a}\OperatorTok{+}\DecValTok{5}
\end{Highlighting}
\end{Shaded}

\begin{verbatim}
## [1] 7
\end{verbatim}

\begin{Shaded}
\begin{Highlighting}[]
\NormalTok{a}\OperatorTok{+}\NormalTok{a}\OperatorTok{^}\DecValTok{2}
\end{Highlighting}
\end{Shaded}

\begin{verbatim}
## [1] 6
\end{verbatim}

\begin{Shaded}
\begin{Highlighting}[]
\KeywordTok{log}\NormalTok{(a)}
\end{Highlighting}
\end{Shaded}

\begin{verbatim}
## [1] 0.6931472
\end{verbatim}

\begin{Shaded}
\begin{Highlighting}[]
\CommentTok{#Ingresar conjunto de datos }

\NormalTok{peso<-}\StringTok{ }\KeywordTok{c}\NormalTok{(}\DecValTok{70}\NormalTok{, }\DecValTok{62}\NormalTok{, }\DecValTok{52}\NormalTok{, }\DecValTok{90}\NormalTok{, }\DecValTok{38}\NormalTok{, }\DecValTok{53}\NormalTok{, }\DecValTok{50}\NormalTok{, }\DecValTok{56}\NormalTok{, }\DecValTok{70}\NormalTok{, }\DecValTok{65}\NormalTok{,}
         \DecValTok{76}\NormalTok{, }\DecValTok{70}\NormalTok{, }\DecValTok{72}\NormalTok{)}
\NormalTok{peso}
\end{Highlighting}
\end{Shaded}

\begin{verbatim}
##  [1] 70 62 52 90 38 53 50 56 70 65 76 70 72
\end{verbatim}

\begin{Shaded}
\begin{Highlighting}[]
\CommentTok{# Descriptivas ------------------------------------------------------------}
\CommentTok{# número de observaciones (length)}

 \KeywordTok{length}\NormalTok{(peso)}
\end{Highlighting}
\end{Shaded}

\begin{verbatim}
## [1] 13
\end{verbatim}

\begin{Shaded}
\begin{Highlighting}[]
 \CommentTok{# calcular la media del peso: sumatoria de las observaciones y dividirlo entre el numero de individuos muestreados}
 
 \KeywordTok{sum}\NormalTok{(peso)}\OperatorTok{/}\NormalTok{(length)(peso)}
\end{Highlighting}
\end{Shaded}

\begin{verbatim}
## [1] 63.38462
\end{verbatim}

\begin{Shaded}
\begin{Highlighting}[]
\NormalTok{ peso.media <-}\StringTok{ }\KeywordTok{sum}\NormalTok{(peso)}\OperatorTok{/}\KeywordTok{length}\NormalTok{(peso)}
 

 \KeywordTok{mean}\NormalTok{(peso)}
\end{Highlighting}
\end{Shaded}

\begin{verbatim}
## [1] 63.38462
\end{verbatim}

\begin{Shaded}
\begin{Highlighting}[]
 \KeywordTok{median}\NormalTok{(peso)}
\end{Highlighting}
\end{Shaded}

\begin{verbatim}
## [1] 65
\end{verbatim}

\begin{Shaded}
\begin{Highlighting}[]
 \KeywordTok{sd}\NormalTok{(peso) }
\end{Highlighting}
\end{Shaded}

\begin{verbatim}
## [1] 13.51874
\end{verbatim}

\begin{Shaded}
\begin{Highlighting}[]
 \KeywordTok{var}\NormalTok{(peso) }
\end{Highlighting}
\end{Shaded}

\begin{verbatim}
## [1] 182.7564
\end{verbatim}

\begin{Shaded}
\begin{Highlighting}[]
 \KeywordTok{fivenum}\NormalTok{(peso)}
\end{Highlighting}
\end{Shaded}

\begin{verbatim}
## [1] 38 53 65 70 90
\end{verbatim}

\begin{Shaded}
\begin{Highlighting}[]
 \KeywordTok{range}\NormalTok{(peso)}
\end{Highlighting}
\end{Shaded}

\begin{verbatim}
## [1] 38 90
\end{verbatim}

\begin{Shaded}
\begin{Highlighting}[]
\CommentTok{# Gráfica -----------------------------------------------------------------}

 \KeywordTok{boxplot}\NormalTok{(peso)}
\end{Highlighting}
\end{Shaded}

\includegraphics{Script_1_files/figure-latex/unnamed-chunk-1-1.pdf}

\begin{Shaded}
\begin{Highlighting}[]
 \KeywordTok{boxplot}\NormalTok{(peso, }\DataTypeTok{col =} \StringTok{"pink"}\NormalTok{, }\DataTypeTok{ylab=} \StringTok{"peso (kg)"}\NormalTok{,}
         \DataTypeTok{main=} \StringTok{"peso alumnos tomado el 29.01.2020"}\NormalTok{)}
\end{Highlighting}
\end{Shaded}

\includegraphics{Script_1_files/figure-latex/unnamed-chunk-1-2.pdf}

\end{document}
